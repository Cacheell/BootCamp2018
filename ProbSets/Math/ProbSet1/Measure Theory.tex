\documentclass[letterpaper,12pt]{article}
\usepackage{array}
\usepackage{threeparttable}
\usepackage{geometry}
\geometry{letterpaper,tmargin=1in,bmargin=1in,lmargin=1.25in,rmargin=1.25in}
\usepackage{fancyhdr,lastpage}
\pagestyle{fancy}
\lhead{}
\chead{}
\rhead{}
\lfoot{}
\cfoot{}
\rfoot{\footnotesize\textsl{Page \thepage\ of \pageref{LastPage}}}
\renewcommand\headrulewidth{0pt}
\renewcommand\footrulewidth{0pt}
\usepackage[format=hang,font=normalsize,labelfont=bf]{caption}
\usepackage{listings}
\lstset{frame=single,
  language=Python,
  showstringspaces=false,
  columns=flexible,
  basicstyle={\small\ttfamily},
  numbers=none,
  breaklines=true,
  breakatwhitespace=true
  tabsize=3
}
\usepackage{amsmath}
\usepackage{amssymb}
\usepackage{amsfonts}
\usepackage{amsthm}
\usepackage{harvard}
\usepackage{setspace}
\usepackage{float,color}
\usepackage[pdftex]{graphicx}
\usepackage{hyperref}
\hypersetup{colorlinks,linkcolor=red,urlcolor=blue}
\theoremstyle{definition}
\newtheorem{theorem}{Theorem}
\newtheorem{acknowledgement}[theorem]{Acknowledgement}
\newtheorem{algorithm}[theorem]{Algorithm}
\newtheorem{axiom}[theorem]{Axiom}
\newtheorem{case}[theorem]{Case}
\newtheorem{claim}[theorem]{Claim}
\newtheorem{conclusion}[theorem]{Conclusion}
\newtheorem{condition}[theorem]{Condition}
\newtheorem{conjecture}[theorem]{Conjecture}
\newtheorem{corollary}[theorem]{Corollary}
\newtheorem{criterion}[theorem]{Criterion}
\newtheorem{definition}[theorem]{Definition}
\newtheorem{derivation}{Derivation} % Number derivations on their own
\newtheorem{example}[theorem]{Example}
\newtheorem{exercise}[theorem]{Exercise}
\newtheorem{lemma}[theorem]{Lemma}
\newtheorem{notation}[theorem]{Notation}
\newtheorem{problem}[theorem]{Problem}
\newtheorem{proposition}{Proposition} % Number propositions on their own
\newtheorem{remark}[theorem]{Remark}
\newtheorem{solution}[theorem]{Solution}
\newtheorem{summary}[theorem]{Summary}
%\numberwithin{equation}{section}
\bibliographystyle{aer}
\newcommand\ve{\varepsilon}
\newcommand\boldline{\arrayrulewidth{1pt}\hline}


\begin{document}

\begin{flushleft}
  \textbf{\large{Problem Set : Measure Theory}} \\
  OSM 2018 Dr. Evans \\
  Cache Ellsworth
  \end{flushleft}

\vspace{5mm}

\noindent\textbf{Exercise 1.3
Which of the these are algebras?  Which are even $\sigma-$algebras?}\\
\indent\textbf{$ G_1 \,)$ }This is not an algebra or a $\sigma-$algebra because the compliment to a finite open set is closed, which is not contained in $G_1$ \\
\indent\textbf{$G_2 \,)$ } This is an algebra but not a $\sigma-$algebra because it is a finite union of intervals. \\
\indent\textbf{$G_3 \,)$ } This is both an algebra and a $\sigma-$algebra because it is a countable union and so therefore could be infinite.  Also if it is a $\sigma-$algebra, it is also an algebra. *********************************************************\\

\vspace{5mm}

\noindent\textbf{Exercise 1.7
Explain why these are the 'largest' and 'smallest' possible $\sigma-$algebras.}\\
\indent\textbf{$\{\emptyset, X\}\,)$} This is the smallest set because by definition of an algebra it must contain the empty set and the compliment.  The compliment of the empty set is the full set.  Therefore the minimum conditions to make an algebra is the set $\{\emptyset, X\}$.\\
\indent \textbf{$\{\mathcal{P}(X)\}\,)$} This is the largest set because ...duh. ***********************************************************\\

\vspace{5mm}

\noindent\textbf{Exercise 1.10
Prove the following:}\\
\begin{proof}  Let $\{ \mathcal{S}_\alpha \}$ be a family of $\sigma-$algebras on $X$.  We show each property of a $\sigma-$algebra.  First, each $\mathcal{S}_\alpha$ has the empty set because they are an algebra, therefore $\cap_\alpha \mathcal{S}_\alpha$ has the empty set.  Second, we let choose an arbitrary set, $A \in \cap_\alpha \mathcal{S}_\alpha$ then $A \in \mathcal{S}_\alpha$ for some $\alpha$. Because each $\mathcal{S}_\alpha$ is an algebra, we know that $A^c \in \cap_\alpha \mathcal{S}_\alpha$.  Third, we choose arbitrary sets $A_1$,$ A_2$, $\dots \in \cap_\alpha \mathcal{S}_\alpha$.  Then each of these sets are in a $\sigma -$ algebra, so therefore $\cup_{i=1}^\infty A_i \in  \cap_\alpha \mathcal{S}_\alpha$.  By these three properties, we see that $\cap_\alpha \mathcal{S}_\alpha$ is also a $\sigma -$ algebra.\\
\end{proof}
\vspace{5mm}

\noindent\textbf{Exercise 1.17
Prove that $\mu$ is monotone and countably sub-additive}\\
\begin{proof}  Let $( X, \mathcal{S}, \mu)$ be a measure space.
Now let $A$, $B$ $\in \mathcal{S}$, and $A \subset B$.  We notice $B=(B\cap A^c)\cup A$, where $(B\cap A^c) \cap A = \emptyset $, so they are disjoint.  By definition $2$ of measurable spaces $\mu (B) = \mu (B\cap A^c) + \mu (A) \ge 0$. Since $\mu(B\cap A^c)  \ge 0$, we know $\mu(A) \leq \mu(B)$.
\end{proof}

\begin{proof}
Let $( X, \mathcal{S}, \mu)$ be a measure space.  Now let $\{ A_i \}_{i=1}^\infty \subset \mathcal{A}$.  We create a disjoint sequence, $B_1 = A_1$, $B_2 = A_2 \cap A_1^c$, $B_3 = A_3 \cap (A_1^c \cup A_2^c)$, $\dots$, $B_i = A_i \smallsetminus (\cup_{j=1}^{i-1} A_j)$. These sets are disjoint and cover the same area as $\cup_{i=1}^\infty A_i$ and so $\mu (\cup_{i=1}^\infty A_i) = \mu (\cup_{i=1}^\infty B_i) = \sum_{i=1}^\infty \mu(B_i)$ by the second property of a measure.  Then by using monotonicity, $\sum_{i=1}^\infty \mu(B_i) \leq \sum_{i=1}^\infty \mu(A_i)$ Thus, $\mu (\cup_{i=1}^\infty A_i) \leq \sum_{i=1}^\infty \mu(A_i)$.
\end{proof}

\vspace{5mm}

\noindent\textbf{Exercise 1.18
Show that $\lambda : \mathcal{S} \rightarrow [0, \infty]$ defined by $\lambda(A) = \mu(A\cap B)$ is also a measure $(X, \mathcal{S})$.}\\
\begin{proof}
Let $(X, \mathcal{S}, \mu)$ be a measure space and let $B \in \mathcal{S}$.  Let $\lambda(A) : \mathcal{S} \rightarrow [0, \infty]$ where $\lambda(A) = \mu(A\cap B)$ .  We show the two properties of a measure.  First, $\lambda(\emptyset) = \mu(\emptyset\cap B) = \mu(\emptyset) = 0$.  Therefore the first property is satisfied.  Secondly, $\lambda(\cup_{i=1}^\infty A_i)=\mu((\cup_{i=1}^\infty A_i)\cap B) = \mu(\cup_{i=1}^\infty(A_i\cup B))=\sum_{i=1}^\infty \mu(A_i\cup B)=\sum_{i=1}^\infty \lambda(A_i)$ because $(A_i\cup B)$ is disjoint due to $A$ being disjoint.\\
\end{proof}

\noindent\textbf{Exercise 1.20
Prove that $( A_1 \supset A_2 \supset A_3 \supset \dots$ , $ A_i \in \mathcal{S}$, $\mu (A_1) \textless \infty ) \Rightarrow (lim_{n\to\infty} \mu(A_n) = \mu(\cap_{i=1}^\infty A_i)) )$}
\begin{proof}
Let $\mu$ be a measure on $(X, \mathcal{S})$ and $( A_1 \supset A_2 \supset A_3 \supset \dots$ , $ A_i \in \mathcal{S}$, $\mu (A_1) \textless \infty )$.  ************************************************************************************
\end{proof}


\noindent\textbf{Exercise 2.10
Explain why $(*)$ in the preceding theorem could be replaced by $\mu^*(B) = \mu^*(B \cap E) + \mu^*(B \cap E^c)$.\\
\indent 




\end{document}