\documentclass[letterpaper,12pt]{article}
\usepackage{array}
\usepackage{threeparttable}
\usepackage{geometry}
\geometry{letterpaper,tmargin=1in,bmargin=1in,lmargin=1.25in,rmargin=1.25in}
\usepackage{fancyhdr,lastpage}
\pagestyle{fancy}
\lhead{}
\chead{}
\rhead{}
\lfoot{}
\cfoot{}
\rfoot{\footnotesize\textsl{Page \thepage\ of \pageref{LastPage}}}
\renewcommand\headrulewidth{0pt}
\renewcommand\footrulewidth{0pt}
\usepackage[format=hang,font=normalsize,labelfont=bf]{caption}
\usepackage{listings}
\usepackage{amsfonts}
\usepackage{mathtools}
\lstset{frame=single,
  language=Python,
  showstringspaces=false,
  columns=flexible,
  basicstyle={\small\ttfamily},
  numbers=none,
  breaklines=true,
  breakatwhitespace=true
  tabsize=3
}
\usepackage{amsmath}
\usepackage{amssymb}
\usepackage{amsfonts}
\usepackage{amsthm}
\usepackage{harvard}
\usepackage{setspace}
\usepackage{float,color}
\usepackage[pdftex]{graphicx}
\usepackage{hyperref}
\hypersetup{colorlinks,linkcolor=red,urlcolor=blue}
\theoremstyle{definition}
\newtheorem{theorem}{Theorem}
\newtheorem{acknowledgement}[theorem]{Acknowledgement}
\newtheorem{algorithm}[theorem]{Algorithm}
\newtheorem{axiom}[theorem]{Axiom}
\newtheorem{case}[theorem]{Case}
\newtheorem{claim}[theorem]{Claim}
\newtheorem{conclusion}[theorem]{Conclusion}
\newtheorem{condition}[theorem]{Condition}
\newtheorem{conjecture}[theorem]{Conjecture}
\newtheorem{corollary}[theorem]{Corollary}
\newtheorem{criterion}[theorem]{Criterion}
\newtheorem{definition}[theorem]{Definition}
\newtheorem{derivation}{Derivation} % Number derivations on their own
\newtheorem{example}[theorem]{Example}
\newtheorem{exercise}[theorem]{Exercise}
\newtheorem{lemma}[theorem]{Lemma}
\newtheorem{notation}[theorem]{Notation}
\newtheorem{problem}[theorem]{Problem}
\newtheorem{proposition}{Proposition} % Number propositions on their own
\newtheorem{remark}[theorem]{Remark}
\newtheorem{solution}[theorem]{Solution}
\newtheorem{summary}[theorem]{Summary}
%\numberwithin{equation}{section}
\bibliographystyle{aer}
\newcommand\ve{\varepsilon}
\newcommand\boldline{\arrayrulewidth{1pt}\hline}


\begin{document}

\begin{flushleft}
  \textbf{\large{Problem Set : Measure Theory}} \\
  OSM 2018 Dr. Evans \\
  Cache Ellsworth
  \end{flushleft}

\vspace{5mm}

\noindent\textbf{Exercise 1.3:
Which of the these are algebras?  Which are even $\sigma-$algebras?}\\
\indent\textbf{$ G_1 \,)$ }This is not an algebra or a $\sigma-$algebra because the compliment to a finite open set is closed, which is not contained in $G_1$ \\
\indent\textbf{$G_2 \,)$ } This is an algebra but not a $\sigma-$algebra because it is a finite union of intervals. It is an algebra because $1)$ it contains the empty set. $2)$\\
\indent\textbf{$G_3 \,)$ } This is both an algebra and a $\sigma-$algebra because it is a countable union and so therefore could be infinite.  Also if it is a $\sigma-$algebra, it is also an algebra. *********************************************************\\

\vspace{5mm}

\noindent\textbf{Exercise 1.7:
Explain why these are the 'largest' and 'smallest' possible $\sigma-$algebras.}\\
\indent\textbf{$\{\emptyset, X\}\,)$} This is the smallest set because by definition of an algebra it must contain the empty set and the compliment.  The compliment of the empty set is the full set.  Therefore the minimum conditions to make an algebra is the set $\{\emptyset, X\}$.\\
\indent \textbf{$\{\mathcal{P}(X)\}\,)$} This is the largest set because it contains every possible combination of subsets of $X$.  Therefore no other combination or it's compliment can be contained. Thus nothing else can be bigger.\\

\vspace{5mm}

\noindent\textbf{Exercise 1.10:
Prove the following:}\\
\begin{proof}  Let $\{ \mathcal{S}_\alpha \}$ be a family of $\sigma-$algebras on $X$.  We show each property of a $\sigma-$algebra.  First, each $\mathcal{S}_\alpha$ has the empty set because they are an algebra, therefore $\cap_\alpha \mathcal{S}_\alpha$ has the empty set.  Second, we let choose an arbitrary set, $A \in \cap_\alpha \mathcal{S}_\alpha$ then $A \in \mathcal{S}_\alpha$ for some $\alpha$. Because each $\mathcal{S}_\alpha$ is an algebra, we know that $A^c \in \cap_\alpha \mathcal{S}_\alpha$.  Third, we choose arbitrary sets $A_1$,$ A_2$, $\dots \in \cap_\alpha \mathcal{S}_\alpha$.  Then each of these sets are in a $\sigma -$ algebra, so therefore $\cup_{i=1}^\infty A_i \in  \cap_\alpha \mathcal{S}_\alpha$.  By these three properties, we see that $\cap_\alpha \mathcal{S}_\alpha$ is also a $\sigma -$ algebra.\\
\end{proof}
\vspace{5mm}

\noindent\textbf{Exercise 1.17:
Prove that $\mu$ is monotone and countably sub-additive}\\
\begin{proof}  Let $( X, \mathcal{S}, \mu)$ be a measure space.
Now let $A$, $B$ $\in \mathcal{S}$, and $A \subset B$.  We notice $B=(B\cap A^c)\cup A$, where $(B\cap A^c) \cap A = \emptyset $, so they are disjoint.  By definition $2$ of measurable spaces $\mu (B) = \mu (B\cap A^c) + \mu (A) \ge 0$. Since $\mu(B\cap A^c)  \ge 0$, we know $\mu(A) \leq \mu(B)$.
\end{proof}

\begin{proof}
Let $( X, \mathcal{S}, \mu)$ be a measure space.  Now let $\{ A_i \}_{i=1}^\infty \subset \mathcal{A}$.  We create a disjoint sequence, $B_1 = A_1$, $B_2 = A_2 \cap A_1^c$, $B_3 = A_3 \cap (A_1^c \cup A_2^c)$, $\dots$, $B_i = A_i \smallsetminus (\cup_{j=1}^{i-1} A_j)$. These sets are disjoint and cover the same area as $\cup_{i=1}^\infty A_i$ and so $\mu (\cup_{i=1}^\infty A_i) = \mu (\cup_{i=1}^\infty B_i) = \sum_{i=1}^\infty \mu(B_i)$ by the second property of a measure.  Then by using monotonicity, $\sum_{i=1}^\infty \mu(B_i) \leq \sum_{i=1}^\infty \mu(A_i)$ Thus, $\mu (\cup_{i=1}^\infty A_i) \leq \sum_{i=1}^\infty \mu(A_i)$.
\end{proof}

\vspace{5mm}

\noindent\textbf{Exercise 1.18:
Show that $\lambda : \mathcal{S} \rightarrow [0, \infty]$ defined by $\lambda(A) = \mu(A\cap B)$ is also a measure $(X, \mathcal{S})$.}\\
\begin{proof}
Let $(X, \mathcal{S}, \mu)$ be a measure space and let $B \in \mathcal{S}$.  Let $\lambda(A) : \mathcal{S} \rightarrow [0, \infty]$ where $\lambda(A) = \mu(A\cap B)$ .  We show the two properties of a measure.  First, $\lambda(\emptyset) = \mu(\emptyset\cap B) = \mu(\emptyset) = 0$.  Therefore the first property is satisfied.  Secondly, $\lambda(\cup_{i=1}^\infty A_i)=\mu((\cup_{i=1}^\infty A_i)\cap B) = \mu(\cup_{i=1}^\infty(A_i\cup B))=\sum_{i=1}^\infty \mu(A_i\cup B)=\sum_{i=1}^\infty \lambda(A_i)$ because $(A_i\cup B)$ is disjoint due to $A$ being disjoint.\\
\end{proof}

\noindent\textbf{Exercise 1.20:
Prove that $( A_1 \supset A_2 \supset A_3 \supset \dots$ , $ A_i \in \mathcal{S}$, $\mu (A_1) \textless \infty ) \Rightarrow (lim_{n\to\infty} \mu(A_n) = \mu(\cap_{i=1}^\infty A_i)) )$}
\begin{proof}
Let $\mu$ be a measure on $(X, \mathcal{S})$ and $( A_1 \subset A_2 \subset A_3 \subset \dots$ , $ A_i \in \mathcal{S}$, $\mu (A_1) \textless \infty )$. Therefore, $(A_i) < \infty$ for each $i \in \mathbb{N}$.  Then the sequence $(A_1 - A_i)_{i\in \mathbb{N}}$, define $A = \cap_{i\in\mathbb{N} }A_i$.  We notice that $\mathrm{lim}_{i\rightarrow\infty}(A_1 - A_i) = A_1 - \mathrm{lim}_{i\rightarrow\infty} A_i = A_1 - A$.  Then $\mu(\cap^\infty_{i=1} A_i) = \mu[A_1 - \cup^\infty_{i=1}(A_1 - A_i)] = \mu(A_1) - \mu(\cup^{\infty}_{i=1}(A_1 - A_i)) = \mu(A_1) - \mathrm{lim}_{i\rightarrow\infty} \mu(A_1 - A_i) = \mu(A_1) - \mathrm{lim}_{i\rightarrow\infty}[\mu(A_1)-\mu(A_i)] = \mathrm{lim}_{i\rightarrow\infty} \mu(A_i)$.  So $\mu(A) = \mathrm{lim}_{n\rightarrow\infty} \mu(A_i)$.
\end{proof}

\vspace{5mm}

\noindent\textbf{ Exercise 2.10: 
Explain why $(*)$ in the preceding theorem could be replaced by $\mu^*(B) = \mu^*(B \cap E) + \mu^*(B \cap E^c)$.}\\
\indent The measure is countably sub-additive and so therefore the measure is greater than or equal to and so is will never be less than.  Explicitly this is: $\mu^*(B) \leq \mu^*((B\cap E)\cup(B\cap E^c)) \leq \mu^*(B \cap E) + \mu^*(B \cap E^c)$. Putting $(*)$ together gets us the equality.  \\

\noindent\textbf{ Exercise 2.14:
Why is it true that the Borel-algebra $\mathcal{B}(\mathbb{R}$ is a subset of $\mathcal{M})$?}
\begin{proof}
We show that $\sigma(\mathcal{A}) = \sigma(\mathcal{O})$.  Then with this fact we use the Carateodory Extension Theorem - Existence because it shows $\sigma(\mathcal{A}) \subset \mathcal{M}$.  First we let $(a,\;b)$ be an arbitrary subset in
 $\sigma(\mathcal{O})$.  We notice $(a, \:b) = \cup_{n \in \mathbb{N}}(a, \: b - \frac{1}{n}]$.  Thus, $\sigma-(\mathcal{O}) 
\subset \sigma(\mathcal{A})$.  Second, we choose an arbitrary set in $\sigma(\mathcal{A})$.  There are four cases.  Case 
$1:$ the set is $(a,\;b)$.  Then $(a,\:b) = \cap_{n \in \mathbb{N}}(a,\: b + \frac{1}{n})$ Then $(a, \:b) \in \sigma(\mathcal{O})$.  
Case $2:$ the set is of the form $(a$, $b]$.  This is equal to $ \cap_{n\in\mathbb{N}}(a$, $b = \frac{1}{n})$. Then $(a, \:b] \in \sigma(\mathcal{O})$. Case$3: (a,\:\infty) = \cup_{n\in\mathbb{N}}(a, \:n)$.Then $(a, \:\infty) \in \sigma(\mathcal{O})$. Case $4: (-\infty,\:b] = \cup_{n\in\mathbb{N}} (-n,\:b]$ Then $(-\infty, \:b] \in \sigma(\mathcal{O})$
\end{proof}

\noindent\textbf{ Exercise 3.1:
Prove that every countable subset of the real line has Lebesgue measure 0.}
\begin{proof}
Let $A$ be an arbitrary countable subset of of the real line.  Then we can order it such as $A = \{ a_1, a_2, a_3, \dots \} = \cup^{\infty}_{n=1} \{ a_n\}$.  Also $\forall \epsilon >  0$, $ \cup^{\infty}_{n=1} \{a_n\}  \subseteq \cup^{\infty}_{n=1}(a_n-\frac{\epsilon}{2^{n+1}}$, $a_n + \frac{\epsilon}{2^{n+1}})$.  The Lebesgue measure $\overline{\mu}(\cup_{n=1}^{\infty} \{a_n\}) < \overline{\mu}(\cup_{n=1}^{\infty} (a_n - \frac{\epsilon}{2^{n+1}}$, $a_n + \frac{\epsilon}{2^{n+1}} )) \leq \sum^{\infty}_{n=1} \overline{\mu} (a_n - \frac{\epsilon}{2^{n+1}}$, $a_n + \frac{\epsilon}{2^{n+1}} ) = \sum^{\infty}_{n=1} (a_n + \frac{\epsilon}{2^{n+1}} - a_n + \frac{\epsilon}{2^{n+1}}) = \sum^{\infty}_{n=1} \frac{\epsilon}{2^n} = \epsilon$.  Thus the measure is $0$.

\end{proof}

\noindent\textbf{ Exercise 3.4:}
\begin{proof}
 These are all equivalent because it is a $\sigma-$ algebra and so the compliments exist.  Therefore $i$ and $iv$ are equivalent by the compliment of the algebra, such as $\{ x \in E : f(x) < \alpha \} = \{ x \in E : f(x) \geq \alpha \}^c$.  Also $ii$ and $iii$ are compliments of each other. $i$ relates to $iii$ by $\{ x \in E : f(x) < \alpha \} = \cup^{\infty}_{n=1} \{ x \in E : f(x) \leq \alpha = \frac{1}{n} \}$ and $\{ x \in E : f(x) \leq \alpha \} = \cap_{n=1}^{\infty} \{ x \in E : f(x) < \alpha - \frac{1}{n} \}$. Thus each are measurable.

\end{proof}

\noindent\textbf{ Exercise 3.7:}
\begin{proof}
Assume $2$ and $4$.  We know that addition and multiplication are continuous  bivariate functions.  Also, the absolute value is a continuous univariate function Therefore $f + g$, $f \times g$, and $|f|$ are measurable functions because they fall under type $4$ as a special type.  For the other two, we only show the maximum function because max$(f, g) = -$min$(f, g)$ and so the minimum function follows.  We show this by letting $f_1 = f$ and $f_2 = g$ and then $f_m = f - 1$ for $m \geq 3$.  Then using $2$ we can that choosing the supremum, we get the max$(f \mathrm{, } g) \:\forall x$.  
\end{proof}

\noindent\textbf{ Exercise 3.14:}
\begin{proof}
Let $f : x \rightarrow \mathbb{R}$.  Then assume $f$ is bounded and $\exists \{s_n\}$.  So $|f| < M$ for some $M$ and for all $x \in X$. Then $x  \in E_i^M$ for some i.  Let $\epsilon > 0$ be arbitrary.  Let $N \in \mathbb{N}$ be a bound for $f$ where $N \geq M$ satisfying $\frac{1}{2^{N}} < \epsilon$.  So for all $x$ and $n \geq N$ we have $ | s_n(x) - f(x)| < \epsilon$. This is uniform convergence.
\end{proof}

\noindent\textbf{ Exercise 4.13: }
\begin{proof}
Let $f$ be measurable, $||f|| < \mathcal{M}$ on $E \subset \mathcal{M}$ and $\mu(E) < \infty$. We know $-M < f(x) < M$ and so using a previous theorem $-M*\mu(E) < \int_E f \; \mathrm{d}\mu < M*\mu(E)$.  Thus $\int f^+ \; \mathrm{d}\mu < M*\mu(E) < \infty$ and $\int_E f^- \; \mathrm{d}\mu < M*\mu(E) < \infty$.  Thus, $\int_E ||f||\; \mathrm{d}\mu < \infty\; \implies\; f \in \mathcal{L}^1(\mu, E)$.
\end{proof}

\noindent\textbf{ Exercise 4.14: }
\begin{proof}
We prove by the contrapositive.  Let $f \in \mathcal{L}^1(\mu\mathrm{, }E)$ and there exists a set $\hat{E}$ where $f$ is infinite on $\hat{E} \subset E$ and $\mu(\hat{E}) > 0$.  Then, $\int_{\hat{E} f \:d\mu} = \infty \implies \int_{\hat{E}} f^+\:d\mu = \infty$.  Because $\hat{E} \subset E$ we use theorem 4.15 (Which doesn't use this exercise, so we can use it) to show that  $\int_{\hat{E}} f^+\:d\mu \leq \int_{E} f^+\:d\mu = \infty$.  By definition then $f \notin \mathcal{L}(\mu,\: E)$.
\end{proof}

\noindent\textbf{ Exercise 4.15: }
\begin{proof}
Let $f$, $g \in \mathcal{L}^1(\mu \mathrm{, } E)$, and $f \leq g$ on $E$.  Then we know that $f^+ - f^- \leq g^+ - g^-$.  Because $f^-$ is positive, then $f^+ \leq g^+ - g^- + f^-$ where both sides are positive because $f^+$ is positive.  Then using Proposition 4.7 we know that $\int_E f^+ \:d\mu \leq \int_E (g^+ - g^- + f^-) \:d\mu$.   By Definition 4.3, we can break up the integral.  Therefore we break up the $f^-$ part of the integral and put it on the left hand side to get: $\int_E f^+ \: d\mu - \int_E f^- \:d\mu \leq \int_E g^+\:d\mu - \int_E g^- \: d\mu$.  And then using Definition 4.3 again we get $\int_E f \: d\mu \leq \int_E g \: d\mu$.
\end{proof}

\noindent\textbf{ Exercise 4.16: }
\begin{proof}
Let $f \in \mathcal{L}^1(\mu, \: E) \mathrm{, } A \in \mathcal{M} \mathrm{, and } A \subset E$.  We know that $\{A \cap E_i\} \leq \{E \cap E_i \} \:\forall i$ so $\mu(A \cap E_i) \leq \mu(E \cap E_i ) \:\forall i$. Let $s(x) = \sum^N_{i=1} c_i\chi E_i$ then $\int_A sd\mu = \sum^N_{i=1} c_i\mu(E\cap E_i) \leq \sum^N_{i=1} c_i\mu(E\cap E_i) = \int_E sd\mu$.  Thus $sup\{\sum^N_{i=1} c_i \: \mu(A \cap E_i) : 0 \leq s \leq ||f|| \mathrm{, \:is \:simple, measurable}\} \leq sup\{\sum^N_{i=1} c_i \: \mu(E \cap E_i) : 0 \leq s \leq ||f|| \mathrm{, s\:is \:simple, measurable}\}$.  Therefore by definition $\int_A ||f||\:d\mu < \int_E ||f||\:d\mu < \infty$.  So $f \in \mathcal{L}^1(\mu \mathrm{, } A)$.
\end{proof}

\noindent\textbf{ Exercise 4.21: }
\begin{proof}
Let $A$, $B$, $\in \mathcal{M}$, $B \subset A$, $\mu(A - B) = 0$, and $f \in \mathcal{L}^1$.  Then by Theorem 4.19, we let $\lambda_1(A) = \int_A f^+d\mu$ and $\lambda_2(A) = \int_A f^-d\mu$.  Therefore, $\int_A f d\mu = \lambda_1(A) - \lambda_2(A)$.  Then, due to subset properties, $ A = (A - B) \cup B$ and $\lambda$ is a measure.  Then $\lambda_1(A) = \lambda_1(A - B) + \lambda_1(B)$ and $\lambda_2(A) = \lambda_2(A - B) + \lambda_2(B)$.  Then $\lambda_1(A) = \lambda_1(B)$ and $\lambda_2(A) = \lambda_2(B) because by Proposition 4.6 and $f$ then $\mu(A - B) = 0$ shows that $\lambda(A - B) = 0$.  Thus, $\int_A fd\mu = \lambda_1(A) - \lambda_2(A) = \lambda_1(B) - \lambda_2(B) = \int_B fd\mu$.   
\end{proof}
\end{document}