\documentclass[11.5pt, letterpaper, bibtotoc,
    tablecaptionabove, figurecaptionabove]{article}


\setlength{\headheight}{10pt}
\setlength{\headsep}{15pt}
\setlength{\topmargin}{-25pt}
\setlength{\topskip}{0in}
\setlength{\textheight}{8.7in}
\setlength{\footskip}{0.3in}
\setlength{\oddsidemargin}{0.0in}
\setlength{\evensidemargin}{0.0in}
\setlength{\textwidth}{6.5in}

\usepackage{setspace}
\setstretch{1.2}
\setlength{\parskip}{5pt}%{6pt}
\setlength{\parindent}{0pt}

\usepackage{subfigure}
\usepackage{subfiles}
\usepackage{graphicx}
\usepackage{epsfig}
\graphicspath{{images/}{../images/}}
\usepackage{amsmath}
\usepackage{amssymb}
\usepackage{amsthm}
\usepackage{enumerate}
\newtheorem{proposition}{Proposition}
\newtheorem{remark}{Remark}
\newtheorem{lemma}{Lemma}
\newtheorem{notation}{Notation}
\newtheorem{corollary}{Corollary}
\newtheorem{remarks}{Remarks}
\newtheorem{examples}{Examples}
\newtheorem{assumption}{Assumption}
\newtheorem{definition}{Definition}
\newcommand{\norm}[1]{\left\lVert#1\right\rVert}
\DeclareMathOperator{\dom}{dom}
\DeclareMathOperator{\ri}{ri}
\DeclareMathOperator{\interior}{int}
\DeclareMathOperator{\essential}{ess}
\DeclareMathOperator{\range}{range}
\DeclareMathOperator{\diag}{diag}
\DeclareMathOperator{\rank}{rank}

\makeatletter
\newcommand{\leqnomode}{\tagsleft@true\let\veqno\@@leqno}
\newcommand{\reqnomode}{\tagsleft@false\let\veqno\@@eqno}
\newcommand{\vertiii}[1]{{\left\vert\kern-0.25ex\left\vert\kern-0.25ex\left\vert#1
\right\vert\kern-0.25ex\right\vert\kern-0.25ex\right\vert}}
\makeatother

\usepackage{bm}

\usepackage[utf8]{inputenc}
\usepackage[english]{babel}
\usepackage{hyperref}
\hypersetup{
    colorlinks=true,
    linkcolor=blue,
    citecolor=red,
}
\usepackage[margin=1in]{geometry}

\begin{document}

\textbf{Cache Ellsworth}

\section*{Linear Constrained Optimization Exercises}

\subsection*{Exercise 8.1 }
See the jupyter notebook in the same folder.

\subsection*{Exercise 8.2}
See the jupyter notebook in the same folder.

\subsection*{Exercise 8.3}
The linear optimization problem in standard form which would maximize Kenny's profit on these two toys.
\begin{align*}
	\text{maximize}\:\:\:\:&\: 4s + 3d \\
	\text{subject to}\:\:\:\:&\: 15s + 10d \leq 1800 \\
	&2s + 2d \leq 350 \\
	&d \leq 200 \\
	&s, d \geq 0\\
\end{align*}


\subsection*{Exercise 8.4}
The linear optimization problem.
\begin{align*}
	\text{minimize}\:\:\:\: &\: 2x_{AB} + 5x_{AD} + 2x_{BD} + 5x_{BC} + 7x_{BE} +9x_{BF} + 4x_{DE} + 3x_{EF} + 2x_{CF}\\
	\text{subject to}\:\:\:\: &\: (x_{AB} + x_{AD}) - 0 = 10 \\
	&(x_{BC} + x_{BF}+ x_{BE}+ x_{BE}) - (x_{AB})= 1\\
	&(x_{CF}) - (x_{BC})= -2 \\
	&(x_{DE}) - (x_{AD})= -3\\
	&(x_{EF}) - (x_{BE}+ x_{DE})= 4\\
	&0- (x_{CF}+ x_{BF}+x_{EF})= -10\\
\end{align*}


\subsection*{Exercise 8.5}
i)
The dictionary process is as follows
\begin{align*}
	\underline{\zeta = \:\:\: 3x_1 + x_2}\\
	w_1 = 15 - x_1 - 3x_2 \\
	w_2 = 18 - 2x_1 - 3x_2 \\
	w_3 = 4 - x_1 + x_2
\end{align*}
where $x_1 = 0, x_2 = 0, w_1 = 15, w_2 = 18, w_3 = 4$
\begin{align*}
	\underline{\zeta = \:\:\: 12 - 3w_3 + 4x_2}\\
	w_1 = 11 - w_3 - 5x_2 \\
	w_2 = 10 + 2w_3 - 5x_2 \\
	x_1 = 4 - w_3 + x_2
\end{align*}
where $x_1 = 4, x_2 = 0, w_1 = 11, w_2 = 10, w_3 = 0$

\begin{align*}
	\underline{\zeta = \:\:\: 20 - \frac{7}{5}w_3 - \frac{4}{5}w_2}\\
	w_1 = 1 - w_3 + w_2 \\
	x_2 = 2 + \frac{2}{5}w_3 - \frac{w_2}{5} \\
	x_1 = 6 - \frac{3}{5}w_3 - \frac{w_2}{5}
\end{align*}
where $x_1 = 6, x_2 = 2, w_1 = 1, w_2 = 0, w_3 = 0$\\
This shows that the optimal point is $(6, 2)$ and the optimum value is 20.  These agree with the same answers I got in Exercise 8.2.
ii)
The dictionary process is as follows
\begin{align*}
	\underline{\zeta = \:\:\: 4x + 6y}\\
	w_1 = 11 + x - y \\
	w_2 = 27 - x - y \\
	w_3 = 90 - 2x - 5y
\end{align*}
where $x = 0, y = 0, w_1 = 11, w_2 = 27, w_3 = 90$
\begin{align*}
	\underline{\zeta = \:\:\: 108 - 4w_2 +2y}\\
	x = 27 - w_2 - y\\
	w_1 = 38 - w_2 - 2y \\
	w_2 = 27 - x - y \\
	w_3 = 36 + 2w_2 - 3y
\end{align*}
where $x = 27, y = 0, w_1 = 38, w_2 = 0, w_3 = 68$
\begin{align*}
	\underline{\zeta = \:\:\: 132 - \frac{8}{3}w_2 - \frac{2}{3}w_3}\\
	x = 15 - \frac{5}{3}w_2 + \frac{w_3}{3}\\
	y = 12 + \frac{2w_2}{3} - \frac{w_3}{3} \\
	w_1 = 14 - \frac{7}{3}w_2 + \frac{2}{3}w_3 \\
\end{align*}
where $x = 15, y = 12, w_1 = 14, w_2 = 0, w_3 = 0$\\
This shows that the optimal point is $(15, 12)$ and the optimum value is 132.  These agree with the same answers I got in Exercise 8.2.

\end{document}