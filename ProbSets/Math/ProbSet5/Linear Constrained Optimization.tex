\documentclass[11.5pt, letterpaper, bibtotoc,
    tablecaptionabove, figurecaptionabove]{article}


\setlength{\headheight}{10pt}
\setlength{\headsep}{15pt}
\setlength{\topmargin}{-25pt}
\setlength{\topskip}{0in}
\setlength{\textheight}{8.7in}
\setlength{\footskip}{0.3in}
\setlength{\oddsidemargin}{0.0in}
\setlength{\evensidemargin}{0.0in}
\setlength{\textwidth}{6.5in}

\usepackage{setspace}
\setstretch{1.2}
\setlength{\parskip}{5pt}%{6pt}
\setlength{\parindent}{0pt}

\usepackage{subfigure}
\usepackage{subfiles}
\usepackage{graphicx}
\usepackage{epsfig}
\graphicspath{{images/}{../images/}}
\usepackage{amsmath}
\usepackage{amssymb}
\usepackage{amsthm}
\usepackage{enumerate}
\newtheorem{proposition}{Proposition}
\newtheorem{remark}{Remark}
\newtheorem{lemma}{Lemma}
\newtheorem{notation}{Notation}
\newtheorem{corollary}{Corollary}
\newtheorem{remarks}{Remarks}
\newtheorem{examples}{Examples}
\newtheorem{assumption}{Assumption}
\newtheorem{definition}{Definition}
\newcommand{\norm}[1]{\left\lVert#1\right\rVert}
\DeclareMathOperator{\dom}{dom}
\DeclareMathOperator{\ri}{ri}
\DeclareMathOperator{\interior}{int}
\DeclareMathOperator{\essential}{ess}
\DeclareMathOperator{\range}{range}
\DeclareMathOperator{\diag}{diag}
\DeclareMathOperator{\rank}{rank}

\makeatletter
\newcommand{\leqnomode}{\tagsleft@true\let\veqno\@@leqno}
\newcommand{\reqnomode}{\tagsleft@false\let\veqno\@@eqno}
\newcommand{\vertiii}[1]{{\left\vert\kern-0.25ex\left\vert\kern-0.25ex\left\vert#1
\right\vert\kern-0.25ex\right\vert\kern-0.25ex\right\vert}}
\makeatother

\usepackage{bm}

\usepackage[utf8]{inputenc}
\usepackage[english]{babel}
\usepackage{hyperref}
\hypersetup{
    colorlinks=true,
    linkcolor=blue,
    citecolor=red,
}
\usepackage[margin=1in]{geometry}

\begin{document}

\textbf{Cache Ellsworth}

\section*{Linear Constrained Optimization Exercises}

\subsection*{Exercise 8.1 }
See the jupyter notebook in the same folder.

\subsection*{Exercise 8.2}
See the jupyter notebook in the same folder.

\subsection*{Exercise 8.3}
The linear optimization problem in standard form which would maximize Kenny's profit on these two toys.
\begin{align*}
	\text{maximize}\:\:\:\:&\: 4s + 3d \\
	\text{subject to}\:\:\:\:&\: 15s + 10d \leq 1800 \\
	&2s + 2d \leq 350 \\
	&d \leq 200 \\
	&s, d \geq 0\\
\end{align*}


\subsection*{Exercise 8.4}
The linear optimization problem.
\begin{align*}
	\text{minimize}\:\:\:\: &\: 2x_{AB} + 5x_{AD} + 2x_{BD} + 5x_{BC} + 7x_{BE} +9x_{BF} + 4x_{DE} + 3x_{EF} + 2x_{CF}\\
	\text{subject to}\:\:\:\: &\: (x_{AB} + x_{AD}) - 0 = 10 \\
	&(x_{BC} + x_{BF}+ x_{BE}+ x_{BE}) - (x_{AB})= 1\\
	&(x_{CF}) - (x_{BC})= -2 \\
	&(x_{DE}) - (x_{AD})= -3\\
	&(x_{EF}) - (x_{BE}+ x_{DE})= 4\\
	&0- (x_{CF}+ x_{BF}+x_{EF})= -10\\
\end{align*}


\subsection*{Exercise 8.5}
i)
The dictionary process is as follows
\begin{align*}
	\underline{\zeta = \:\:\: 3x_1 + x_2}\\
	w_1 = 15 - x_1 - 3x_2 \\
	w_2 = 18 - 2x_1 - 3x_2 \\
	w_3 = 4 - x_1 + x_2
\end{align*}
where $x_1 = 0, x_2 = 0, w_1 = 15, w_2 = 18, w_3 = 4$
\begin{align*}
	\underline{\zeta = \:\:\: 12 - 3w_3 + 4x_2}\\
	w_1 = 11 - w_3 - 5x_2 \\
	w_2 = 10 + 2w_3 - 5x_2 \\
	x_1 = 4 - w_3 + x_2
\end{align*}
where $x_1 = 4, x_2 = 0, w_1 = 11, w_2 = 10, w_3 = 0$

\begin{align*}
	\underline{\zeta = \:\:\: 20 - \frac{7}{5}w_3 - \frac{4}{5}w_2}\\
	w_1 = 1 - w_3 + w_2 \\
	x_2 = 2 + \frac{2}{5}w_3 - \frac{w_2}{5} \\
	x_1 = 6 - \frac{3}{5}w_3 - \frac{w_2}{5}
\end{align*}
where $x_1 = 6, x_2 = 2, w_1 = 1, w_2 = 0, w_3 = 0$\\
This shows that the optimal point is $(6, 2)$ and the optimum value is 20.  These agree with the same answers I got in Exercise 8.2.
ii)
The dictionary process is as follows
\begin{align*}
	\underline{\zeta = \:\:\: 4x + 6y}\\
	w_1 = 11 + x - y \\
	w_2 = 27 - x - y \\
	w_3 = 90 - 2x - 5y
\end{align*}
where $x = 0, y = 0, w_1 = 11, w_2 = 27, w_3 = 90$
\begin{align*}
	\underline{\zeta = \:\:\: 108 - 4w_2 +2y}\\
	x = 27 - w_2 - y\\
	w_1 = 38 - w_2 - 2y \\
	w_2 = 27 - x - y \\
	w_3 = 36 + 2w_2 - 3y
\end{align*}
where $x = 27, y = 0, w_1 = 38, w_2 = 0, w_3 = 68$
\begin{align*}
	\underline{\zeta = \:\:\: 132 - \frac{8}{3}w_2 - \frac{2}{3}w_3}\\
	x = 15 - \frac{5}{3}w_2 + \frac{w_3}{3}\\
	y = 12 + \frac{2w_2}{3} - \frac{w_3}{3} \\
	w_1 = 14 - \frac{7}{3}w_2 + \frac{2}{3}w_3 \\
\end{align*}
where $x = 15, y = 12, w_1 = 14, w_2 = 0, w_3 = 0$\\
This shows that the optimal point is $(15, 12)$ and the optimum value is 132.  These agree with the same answers I got in Exercise 8.2.

\subsection*{Exercise 8.6}
The dictionary process is as follows
\begin{align*}
	\underline{\zeta = \:\:\: 4s + 3d}\\
	w_1 = 1800 -10d - 15s \\
	w_2 = 350 - 2d - 2s \\
	w_3 = 200 - d
\end{align*}
where $s = 0, d = 0, w_1 = 1800, w_2 = 350, w_3 = 200$
\begin{align*}
	\underline{\zeta = \:\:\: 480 + \frac{1}{3}d - \frac{4}{15}w_1}\\
	s = 120 - \frac{2}{3}d - \frac{w_1}{15}\\
	w_2 = 110 - \frac{2}{3}d + \frac{2}{15}w_1 \\
	w_3 = 200 - d
\end{align*}
where $s = 120, d = 0, w_1 = 0, w_2 = 110, w_3 = 200$
\begin{align*}
	\underline{\zeta = \:\:\: 55 - \frac{1}{2}w_2 - \frac{1}{5}w_1}\\
	s = 10 + w_2 - \frac{w_1}{5}\\
	d = 165 - \frac{3}{2}w_2 + \frac{1}{5}w_1 \\
	w_3 = 35 - \frac{3}{2}w_2 + \frac{1}{5}w_1
\end{align*}
where $s = 10, d = 165, w_1 = 0, w_2 = 0, w_3 = 35$ \\
This shows that the optimal point is $(s=10, d=165)$ and the optimum value is 55. 

\subsection*{Exercise 8.7}
i) 
The dictionary process is as follows
\begin{align*}
	\underline{\zeta = \:\:\: -x_0}\\
	x_3 = -8 + 4x_1 + 2x_2 + x_0 \\
	x_4 = 6 + 2x_1 - 3x_2 + x_0 \\
	x_5 = 3 - x_1 + x_0
\end{align*}
where $x_0 = 0, x_1 = 0, x_2 = 0, x_3 = -8, x_4 = 6, x_5 = 3$ 

\begin{align*}
	\underline{\zeta = \:\:\: -8 + 4x_1 + 2x_2 - x_3}\\
	x_0 = 8 - 4x_1 - 2x_2 + x_3 \\
	x_4 = 14 - 2x_1 - 5x_2 + x_3 \\
	x_5 = 11 - 5x_1 -2x_2 + x_3
\end{align*}
where $x_0 = 8, x_1 = 0, x_2 = 0, x_3 = 0, x_4 = 14, x_5 = 11$

\begin{align*}
	\underline{\zeta = \:\:\: -x0}\\
	x_1 = 2 - \frac{1}{2}x_2 + \frac{1}{4}x_3 - \frac{1}{4}x_0 \\
	x_4 = 10 - 4x_2 + \frac{1}{2}x_3 +\frac{1}{2} x_0 \\
	x_5 = 1 + \frac{1}{2}x_2 - \frac{1}{4}x_3 + \frac{5}{4}x_0 
\end{align*}
where $x_0 = 0, x_1 = 2, x_2 = 0, x_3 = 0, x_4 = 10, x_5 = 1$   \\
This is a feasible point.
\begin{align*}
	\underline{\zeta = \:\:\: 2 + \frac{3}{2}x_2  + \frac{1}{4}x_3}\\
	x_1 = 2 - \frac{1}{2}x_2 + \frac{1}{4}x_3 \\
	x_4 = 10 - 4x_2 + \frac{1}{2}x_3  \\
	x_5 = 1 + \frac{1}{2}x_2 - \frac{1}{4}x_3 
\end{align*}
where $ x_1 = 2, x_2 = 0, x_3 = 0, x_4 = 10, x_5 = 1$   \\

\begin{align*}
	\underline{\zeta = \:\:\: 3 + 2x_2 - x_5}\\
	x_1 = 3 - x_5\\
	x_4 = 12 - 3x_2 - 2x_5\\
	x_3 = 4 + 2x_2 - 4x_5
\end{align*}
where $x_1 = 3, x_2 = 0, x_3 = 4, x_4 = 12, x_5 = 0$   \\

\begin{align*}
	\underline{\zeta = \:\:\: 11 - \frac{2}{3}x_4 - \frac{7}{3}x_5}\\
	x_1 = 3 - x_5\\
	x_2 = 4 - \frac{1}{3}x_4 - \frac{2}{3}x_5 \\
	x_3 = 4 - \frac{8}{3}x_4 - \frac{16}{3}x_5
\end{align*}
where $ x_1 = 3, x_2 = 4, x_3 = 4, x_4 = 0, x_5 = 0$   \\
So the optimal point is $(x_1 = 3, x_2 = 4)$ where the optimal point is $11$.  \\
ii)
The dictionary  process is as follows changing it to the auxiliary problem
\begin{align*}
	\underline{\zeta = \:\:\: -x_0}\\
	x_3 = 15 - 5x_1 - 3x_2 + x_0\\
	x_4 = 15 - 3x_1 - 5x_2 + x_0 \\
	x_5 = -12 - 4x_1 + 3x_2 + x_0
\end{align*}
where $x_0 = 0, x_1 = 15, x_2 = 0, x_3 = 15, x_4 = 15, x_5 = -12$   \\
\begin{align*}
	\underline{\zeta = \:\:\: -12 - 4x_1 + 3x_2 - x_5}\\
	x_3 = 27 - x_1 - 6x_2 + x_5\\
	x_4 = 27 + x_1 - 8x_2 + x_5 \\
	x_0 = 12 + 4x_1 - 3x_2 + x_5
\end{align*}
where $x_0 = 12, x_1 = 0, x_2 = 0, x_3 = 27, x_4 = 27, x_5 = 0$   \\

\begin{align*}
	\underline{\zeta = \:\:\: -\frac{15}{8} - \frac{29}{8}x_1 - \frac{3}{8}x_4 - \frac{5}{8}x_5}\\
	x_3 = \frac{27}{4} - \frac{7}{4}x_1 + \frac{3}{4}x_4 + \frac{1}{4}x_5\\
	x_2 = \frac{27}{8} + \frac{1}{8}x_1 - \frac{1}{8}x_4 + \frac{1}{8}x_5 \\
	x_0 = \frac{15}{8} + \frac{29}{8}x_1 + \frac{3}{8}x_4 + \frac{5}{8}x_5
\end{align*}
where $x_0 = \frac{15}{8}, x_1 = 0, x_2 = \frac{27}{8}, x_3 = \frac{27}{4}, x_4 = 0, x_5 = 0$   \\
This problem has no feasible solutions because the objective function is negative.

iii)
The dictionary process is as follows
\begin{align*}
	\underline{\zeta = \:\:\: -3x_1 + x_2}\\
	w_1 = 4 -x_2\\
	w_2 = 6 + 2x_1 - 3x_2 \\
\end{align*}
where $x_1 = 0, x_2 = 0, w_1 = 4, w_2 = 6$   \\
\begin{align*}
	\underline{\zeta = \:\:\: 2 - \frac{7}{3}x_1 - \frac{1}{3}w_2}\\
	w_1= 2 - x_1 + \frac{1}{3}w_2\\
	x_2 = 2 + x_1 - \frac{1}{3}w_2\\
\end{align*}
Thus the optimal point is $(x_1=0, x_2=2)$ where the optimal value is 2.  

\subsection*{Exercise 8.8}
\begin{align*}
	\text{maximize  } &5z + x + y \\
	& z \leq 4\\
	& z - x \leq 3 \\
	& x, y, z \geq 0\\
\end{align*}
In this example, y is clearly unbounded. So the region is closed and unbounded.  Also there is a unique feasible maximizer at $[1, 0, 4]^T$.

\subsection*{Exercise 8.9}
\begin{align*}
	\text{maximize  } &x_1 + x_2 + x_3 \\
	& x_1 \leq 1\\
	& x_2 \leq 1\\
	& x_1, x_2, x_3 \geq 0
\end{align*}

\subsection*{Exercise 8.10}
\begin{align*}
	\text{maximize  } &x_1 + x_2 + x_3 \\
	& -x_1 \leq -2\\
	& x_1 \leq 1\\
	& x_1, x_2, x_3 \geq 0
\end{align*}

\subsection*{Exercise 8.11}
\begin{align*}
	\text{maximize  } &x_1 + x_2 + x_3 \\
	& x_3 - x_2 - x_1 \leq -1\\
	& x_1 \leq 1\\
	& x_2 \leq 1\\
	& x_3 \leq 1\\
	& x_1, x_2, x_3 \geq 0
\end{align*}

 \subsection*{Exercise 8.12}
 The dictionary process is as follows:
\begin{align*}
	\underline{\zeta = \:\:\: 10x_1 - 57x_2 - 9x_3 - 24x_4}\\
	w_1 = -0.5x_1 + 1.5x_2 + 0.5x_3 - x_4\\
	w_2 = -0.5x_1 + 5.5x_2 + 2.5x_3 - 9x_4 \\
	w_3 = 1 - x_1
\end{align*}
where $x_1 = 0, x_2 = 0, x_3 = 0, x_4 = 0, w_1 = 0, w_2 = 0, w_3 = 1$.
\begin{align*}
	\underline{\zeta = \:\:\: -27x_2 + x_3 - 44x_4 - 20w_1}\\
	x_1 = 3x_2 + x_3 - 2x_4 - 2w_1\\
	w_2 = 4x_2 + 2x_3 - 8x_4 + x_5\\
	w_3 = 1 - 3x_2 - x_3 + 2x_4 + 2w_1
\end{align*}
 where $x_1 = 0, x_2 = 0, x_3 = 0, x_4 = 0, w_1 = 0, w_2 = 0, w_3 = 1$.
\begin{align*}
	\underline{\zeta = \:\:\: 1 - 30x_2 - 42x_4 - 18w_1 - w_3}\\
	x_1 = 1 - w_3\\
	w_2 = 2 - 2x_2 - 4x_4 + 5w_1 - 2w_3\\
	x_3 = 1 - 3x_2  + 2x_4 + 2w_1 - w_3
\end{align*}
 where $x_1 = 1, x_2 = 0, x_3 = 1, x_4 = 0, w_1 = 0, w_2 = 2, w_3 = 0$.\\
 So the optimal point is $(1, 0, 1, 0)$ with an optimal value of 1.

 \subsection*{Exercise 8.15}
 \begin{proof}
 	Let $x \in \mathrm{R}^n$ be a primal feasible point and $y \in mathrm{R}^m$ is a feasible point for the dual problem.  We know that $ Ax \leq b$ and $A^Ty \leq c$.  Applying the transpose of the first equation.
	\begin{align*}
		x^TA^T \leq b^T.  \implies x^TA^Ty \leq x^Tc \leq y^Tb \implies c^Tx \leq b^Ty.
	\end{align*}
 \end{proof}

 \subsection*{Exercise 8.17}
 \begin{proof}
	We change the dual form into the standard form:
	\begin{align*}
	\text{maximize  }& (-b^T)y \\
	\text{subject to  }& (-A)^Ty  \preceq -c \\
	& y \succeq  0
	\end{align*}
	Then the dual of this version (The dual of the dual) is:
	\begin{align*}
		\text{minimize  }& (-c^T)x \\
		\text{subject to  }& ((-A)^T)^Tx  \succeq -b \\
		& x \succeq  0
	\end{align*}
	Changing this into the standard form.
	\begin{align*}
		\text{maximize  }& c^Tx \\
		\text{subject to  }& Ax  \succeq b \\
		& x \succeq  0
	\end{align*}
	This is the original primal problem 
	
\end{proof}
 
 \subsection*{Exercise 8.18}
 The dictionary process is as follows
\begin{align*}
	\underline{\zeta = \:\:\: x_1+ x_2}\\
	w_1 = 3-2x_1 - x_2 \\
	w_2 = 5 - x_1 - 3x_2\\
	w_3 = 4 - 2x_1 - 3x_2
\end{align*}
where $x_1 = 0, x_2 = 0, w_1 = 3, w_2 = 5, w_3 = 4$
\begin{align*}
	\underline{\zeta = \:\:\: \frac{3}{2} + \frac{1}{2}x_2 - \frac{1}{2}w_1}\\
	x_1 = \frac{3}{2} - \frac{1}{2}x_2 - \frac{1}{2}w_1 \\
	w_2 = \frac{7}{2} - \frac{5}{2}x_2  + \frac{1}{2}w_1\\
	w_3 = 1 - 2x_2 + w_1
\end{align*}
where $x_1 = \frac{3}{2}, x_2 = 0, w_1 = 0, w_2 = \frac{7}{2}, w_3 = 1$
\begin{align*}
	\underline{\zeta = \:\:\: \frac{7}{4} - \frac{1}{4}w_1 - \frac{1}{4}w_3}\\ 
	x_1 = \frac{3}{2} - \frac{1}{2}x_2 - \frac{1}{2}w_1 \\
	w_2 = \frac{7}{2} - \frac{5}{2}x_2  + \frac{1}{2}w_1\\
	x_2 = \frac{1}{2} + \frac{1}{2}w_1 - \frac{1}{2}w_3
\end{align*}
where $x_1 = \frac{3}{2}, x_2 = \frac{1}{2}, w_1 = 0, w_2 = \frac{7}{2}, w_3 = 0$
This shows that the optimal point is $(x_1=\frac{5}{4}, x_2=\frac{1}{2})$ and the optimum value is $\frac{7}{4}$. \\

The Dual problem's dictionary process.
\begin{align*}
	\text{minimize  } &3y_1 + 5y_2 + 4y_3\\
	\text{subject to } &2y_1 + y_2 + 2y_3 \geq 1 \\
	& y_1 + 3y_2 +3y_3 \geq 1\\
	& y_1 + 3y_2 +3y_3 \geq 1 \\
	&y_1, y_2, y_3 \geq 0
\end{align*}
Dictionary process

\begin{align*}
	\underline{\zeta = \:\:\: - v_0}\\
	v_1 = -1 +2y_1 +y_2 + 2y_3 + v_0\\
	v_2 = -1 + y_1 + 3y_2 + 3y_3 +v_0\\
\end{align*}
where $v_0 = 0, v_1 = -1, v_2=-1, y_1=0, y_2=0, y_3=0$


\begin{align*}
	\underline{\zeta = \:\:\: -1 +2y_1 +y_2 +2y_3 -v_1}\\
	v_0 = 1 -2y_1 -y_2 -2y_3 +v_1\\
	v_2 = 2 -5y_1 -3y_3 +3v_1 -2v_0\\
\end{align*}
where $v_0 = 1, v_1 = 0, v_2= 2, y_1=0, y_2=0, y_3=0$ 
 
 \begin{align*}
	\underline{\zeta = \:\:\: -v_0}\\
	y_2 = 1 -2y_1 -2y_3 +v_1 - v_0\\
	v_2 = 2 - 5y_1 -3y_3 + 3v_1 - 2v_0\\
\end{align*}
where $v_0 = 0, v_1 = 0, v_2= 2, y_1=0, y_2=1, y_3=0$ 

 \begin{align*}
	\underline{\zeta = \:\:\: -2 + y_1 -3y_2 - 2v_1}\\
	y_3 = \frac{1}{2} - y_1 - \frac{1}{2}y_2 + \frac{1}{2}v_1\\
	v_2 = \frac{1}{2} - 2y_1 + \frac{3}{2}y_2 + \frac{3}{2}v_1\\
\end{align*}
where $v_0 = 0, v_1 = 0, v_2= \frac{1}{2}, y_1=0, y_2=0, y_3=\frac{1}{2}$  

 \begin{align*}
	\underline{\zeta = \:\:\: -\frac{7}{4} - \frac{3}{2}y_2 - \frac{5}{4}v_1 - \frac{1}{2}v_2}\\
	y_3 = \frac{1}{4} - y_1 - \frac{7}{2}y_2 - \frac{1}{4}v_1 + \frac{1}{2}v_2\\
	y_1 = \frac{1}{4} + \frac{3}{2}y_2 + \frac{3}{4}v_1 - \frac{1}{2}v_2\\
\end{align*}
where $v_0 = 0, v_1 = 0, v_2= 0, y_1=\frac{1}{4}, y_2=0, y_3=\frac{1}{4}$  \\
This shows that the optimal point is $(y_1=\frac{1}{4}, y_2 = 0, y_3=\frac{1}{4})$ and the optimum value is $-\frac{7}{4}$. \\
Changing this to the primal form we get the same optimum value.
 
\end{document}